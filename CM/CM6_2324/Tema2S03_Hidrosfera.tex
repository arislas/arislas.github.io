\section{La hidrosfera: las aguas del planeta}

\subsection{La hidrosfera y su distribución}

La \textbf{hidrosfera} es el conjunto de todas las aguas de la Tierra: océanos, mares, ríos, lagos, glaciares, aguas subterráneas y vapor de agua. Cubre, aproximadamente, el 70 \% de la superficie terrestre. El agua es una sustancia que puede encontrarse en la naturaleza en tres estados: líquido, hielo y vapor de agua. El agua en la Tierra está distribuida de la siguiente forma: el 97,5 \% es salada (mares y océanos) y solo el 2,5 \% es dulce.

\subsection{La hidrosfera y su clasificación}

Las aguas de la hidrosfera están presentes en dos conjuntos: aguas superficiales y aguas subterráneas.

\vspace{3mm}
\textbf{Las aguas superficiales}

\vspace{3mm}
Estas aguas se encuentran sobre la superficie de la corteza terrestre. Se distinguen dos grandes grupos:
\begin{itemize}
    \item \textbf{Aguas marinas}. Están formadas por océanos y mares. Son saladas y cubren gran parte de la superficie terrestre.
    \item \textbf{Aguas continentales}. A este grupo pertenecen ríos, arroyos, lagos, lagunas, masas de hielo de los polos, nieve... Suelen ser dulces.
\end{itemize}

\textbf{Las aguas subterráneas}

\vspace{3mm}
Se forman cuando las aguas superficiales y el agua de lluvia se filtran a través del suelo y se almacenan en el interior de la Tierra en depósitos denominados acuíferos. A veces, cuando circulan bajo la tierra, forman cuevas y galerías. En algunos lugares salen a la superficie, en forma de fuentes o manantiales. En otras ocasiones, se accede a ellas mediante la excavación de pozos.

\subsection{La importancia de la hidrosfera}

El agua de la hidrosfera es fundamental para el ser humano y el resto de los seres vivos, ya que:
\begin{itemize}
    \item Contribuye a mantener templada la Tierra porque absorbe el calor del Sol.
    \item Es el medio en el que viven los seres acuáticos.
    \item Los seres vivos la necesitamos: las plantas, para fabricar su alimento en la fotosíntesis; los animales, para el funcionamiento del cuerpo.
    \item Los seres humanos la utilizamos para las industrias, la limpieza, el ocio, los transportes, etcétera.
\end{itemize}
\batchmode
\documentclass[12pt,a4paper]{book}
\RequirePackage{ifthen}


\usepackage{graphicx} 
\usepackage[utf8]{inputenc}                                          
\usepackage[spanish]{babel}                                          
\usepackage{times}                                                   
\usepackage[left=20mm,top=20mm,bottom=20mm,right=20mm]{geometry}     
\usepackage{amsmath,amssymb,yhmath}
\usepackage{xlop}                                                    
\usepackage[usenames]{color}
\usepackage{eurosym}
\usepackage{multicol}
\usepackage{cancel}
\usepackage{enumerate}                                               
\usepackage{emptypage}                                               
\usepackage{pstricks}                                                
\usepackage{polynom}                                                 
\usepackage{venndiagram}                                             
\usepackage{tikz}                                                    
\usetikzlibrary{positioning,shapes,fit,arrows,babel}                 %Librería tikz
\usepackage{framed}\definecolor{shadecolor}{RGB}{224,238,238}%
\providecommand{\placeholder}[1]{\rule{3mm}{0.1mm}}                       %Print -- regardless of the input%
\providecommand{\gobble}[1]{}                                             %Print <nothing> regardless of the input
\usepackage{verbatim}                                                
\usepackage{multirow}                                                
\usepackage[hidelinks]{hyperref}                                     
\usepackage{array}                                                   
\usepackage{vwcol} 
\usetikzlibrary{datavisualization}
\usepackage{tkz-euclide}
\usepackage{bm} 

%
\providecommand{\abs}[1]{\lvert#1\rvert} % para el valor absoluto%
\providecommand{\norm}[1]{\lVert#1\rVert} % Para el módulo

\usepackage{tcolorbox}
\tcbuselibrary{listingsutf8}

\newtcolorbox[auto counter, number within=subsubsection]{ejemplos}[2][]{colback=green!5!white,colframe=green!75!black, fonttitle=\bfseries , title=Ejemplo~\thetcbcounter: #2,#1}

\newtcolorbox[auto counter, number within=subsubsection]{sintaxis}[2][]{colback=cyan!5!white,colframe=cyan!75!black, fonttitle=\bfseries , title=Sintaxis~\thetcbcounter: #2,#1}

\newtcolorbox[auto counter, number within=subsubsection]{nota}[2][]{colback=pink!5!white,colframe=pink!75!black, fonttitle=\bfseries , title=Nota~\thetcbcounter: #2,#1}

\newtcolorbox{ejemplo}[2][]
{colback=green!5!white,colframe=green!75!black,
fonttitle=\bfseries , title=Ejemplos}

\usepackage{listings}

\setcounter{secnumdepth}{3}                                          % para que ponga 1.1.1.1 en subsubsecciones. El número 3 es el número de sub secciones que utilizará
\setcounter{tocdepth}{3}                                             % para que ponga subsubsecciones en el indice. El número 3 es el número de sub secciones que utilizará

\title{CONOCIMIENTO DEL MEDIO 6}
\author{}
\date{}

\setlength{\parindent}{0cm}                                           %Elimina sangrado
\clearpage                                                           %Elimina numeración páginas en blanco





\makeatletter

\makeatletter
\count@=\the\catcode`\_ \catcode`\_=8 
\newenvironment{tex2html_wrap}{}{}%
\catcode`\<=12\catcode`\_=\count@
\newcommand{\providedcommand}[1]{\expandafter\providecommand\csname #1\endcsname}%
\newcommand{\renewedcommand}[1]{\expandafter\providecommand\csname #1\endcsname{}%
  \expandafter\renewcommand\csname #1\endcsname}%
\newcommand{\newedenvironment}[1]{\newenvironment{#1}{}{}\renewenvironment{#1}}%
\let\newedcommand\renewedcommand
\let\renewedenvironment\newedenvironment
\makeatother
\let\mathon=$
\let\mathoff=$
\ifx\AtBeginDocument\undefined \newcommand{\AtBeginDocument}[1]{}\fi
\newbox\sizebox
\setlength{\hoffset}{0pt}\setlength{\voffset}{0pt}
\addtolength{\textheight}{\footskip}\setlength{\footskip}{0pt}
\addtolength{\textheight}{\topmargin}\setlength{\topmargin}{0pt}
\addtolength{\textheight}{\headheight}\setlength{\headheight}{0pt}
\addtolength{\textheight}{\headsep}\setlength{\headsep}{0pt}
\newwrite\lthtmlwrite
\makeatletter
\let\realnormalsize=\normalsize
\global\topskip=2sp
\def\preveqno{}\let\real@float=\@float \let\realend@float=\end@float
\def\@float{\let\@savefreelist\@freelist\real@float}
\def\liih@math{\ifmmode$\else\bad@math\fi}
\def\end@float{\realend@float\global\let\@freelist\@savefreelist}
\let\real@dbflt=\@dbflt \let\end@dblfloat=\end@float
\let\@largefloatcheck=\relax
\let\if@boxedmulticols=\iftrue
\def\@dbflt{\let\@savefreelist\@freelist\real@dbflt}
\def\adjustnormalsize{\def\normalsize{\mathsurround=0pt \realnormalsize
 \parindent=0pt\abovedisplayskip=0pt\belowdisplayskip=0pt}%
 \def\phantompar{\csname par\endcsname}\normalsize}%
\def\lthtmltypeout#1{{\let\protect\string \immediate\write\lthtmlwrite{#1}}}%
\usepackage[tightpage,active]{preview}
\PreviewBorder=0.5bp
\newbox\lthtmlPageBox
\newdimen\lthtmlCropMarkHeight
\newdimen\lthtmlCropMarkDepth
\long\def\lthtmlTightVBoxA#1{\def\lthtmllabel{#1}
    \setbox\lthtmlPageBox\vbox\bgroup\catcode`\_=8 }%
\long\def\lthtmlTightVBoxZ{\egroup
    \lthtmlCropMarkHeight=\ht\lthtmlPageBox \advance \lthtmlCropMarkHeight 6pt
    \lthtmlCropMarkDepth=\dp\lthtmlPageBox
    \lthtmltypeout{^^J:\lthtmllabel:lthtmlCropMarkHeight:=\the\lthtmlCropMarkHeight}%
    \lthtmltypeout{^^J:\lthtmllabel:lthtmlCropMarkDepth:=\the\lthtmlCropMarkDepth:1ex:=\the \dimexpr 1ex}%
    \begin{preview}\copy\lthtmlPageBox\end{preview}}%
\long\def\lthtmlTightFBoxA#1{\def\lthtmllabel{#1}%
    \adjustnormalsize\setbox\lthtmlPageBox=\vbox\bgroup\hbox\bgroup %
    \let\ifinner=\iffalse \let\)\liih@math %
    \bgroup\catcode`\_=8 }%
\long\def\lthtmlTightFBoxZ{\egroup\egroup
    \@next\next\@currlist{}{\def\next{\voidb@x}}%
    \expandafter\box\next\egroup %
    \lthtmlCropMarkHeight=\ht\lthtmlPageBox \advance \lthtmlCropMarkHeight 6pt
    \lthtmlCropMarkDepth=\dp\lthtmlPageBox
    \lthtmltypeout{^^J:\lthtmllabel:lthtmlCropMarkHeight:=\the\lthtmlCropMarkHeight}%
    \lthtmltypeout{^^J:\lthtmllabel:lthtmlCropMarkDepth:=\the\lthtmlCropMarkDepth:1ex:=\the \dimexpr 1ex}%
    \begin{preview}\copy\lthtmlPageBox\end{preview}}%
    \long\def\lthtmlinlinemathA#1#2\lthtmlindisplaymathZ{\lthtmlTightVBoxA{#1}{\hbox\bgroup#2\egroup}\lthtmlTightVBoxZ}
    \def\lthtmlinlineA#1#2\lthtmlinlineZ{\lthtmlTightVBoxA{#1}{\hbox\bgroup#2\egroup}\lthtmlTightVBoxZ}
    \long\def\lthtmldisplayA#1#2\lthtmldisplayZ{\lthtmlTightVBoxA{#1}{#2}\lthtmlTightVBoxZ}
    \long\def\lthtmldisplayB#1#2\lthtmldisplayZ{\\edef\preveqno{(\theequation)}%
        \lthtmlTightVBoxA{#1}{\let\@eqnnum\relax#2}\lthtmlTightVBoxZ}
    \long\def\lthtmlfigureA#1{\let\@savefreelist\@freelist
        \lthtmlTightFBoxA{#1}}
    \long\def\lthtmlfigureZ{
        \lthtmlTightFBoxZ\global\let\@freelist\@savefreelist}
    \long\def\lthtmlpictureA#1{\let\@savefreelist\@freelist
        \lthtmlTightVBoxA{#1}}
    \long\def\lthtmlpictureZ{
        \lthtmlTightVBoxZ\global\let\@freelist\@savefreelist}
\def\lthtmlcheckvsize{\ifdim\ht\sizebox<\vsize 
  \ifdim\wd\sizebox<\hsize\expandafter\hfill\fi \expandafter\vfill
  \else\expandafter\vss\fi}%
\providecommand{\selectlanguage}[1]{}%
\makeatletter \tracingstats = 1 
\providecommand{\Alpha}{\textrm{A}}
\providecommand{\Beta}{\textrm{B}}
\providecommand{\Chi}{\textrm{X}}
\providecommand{\Epsilon}{\textrm{E}}
\providecommand{\Eta}{\textrm{H}}
\providecommand{\Iota}{\textrm{J}}
\providecommand{\Kappa}{\textrm{K}}
\providecommand{\Mu}{\textrm{M}}
\providecommand{\Nu}{\textrm{N}}
\providecommand{\Omicron}{\textrm{O}}
\providecommand{\Rho}{\textrm{R}}
\providecommand{\Tau}{\textrm{T}}
\providecommand{\Zeta}{\textrm{Z}}
\providecommand{\omicron}{\textrm{o}}


\begin{document}
\pagestyle{empty}\thispagestyle{empty}\lthtmltypeout{}%
\lthtmltypeout{latex2htmlLength hsize=\the\hsize}\lthtmltypeout{}%
\lthtmltypeout{latex2htmlLength vsize=\the\vsize}\lthtmltypeout{}%
\lthtmltypeout{latex2htmlLength hoffset=\the\hoffset}\lthtmltypeout{}%
\lthtmltypeout{latex2htmlLength voffset=\the\voffset}\lthtmltypeout{}%
\lthtmltypeout{latex2htmlLength topmargin=\the\topmargin}\lthtmltypeout{}%
\lthtmltypeout{latex2htmlLength topskip=\the\topskip}\lthtmltypeout{}%
\lthtmltypeout{latex2htmlLength headheight=\the\headheight}\lthtmltypeout{}%
\lthtmltypeout{latex2htmlLength headsep=\the\headsep}\lthtmltypeout{}%
\lthtmltypeout{latex2htmlLength parskip=\the\parskip}\lthtmltypeout{}%
\lthtmltypeout{latex2htmlLength oddsidemargin=\the\oddsidemargin}\lthtmltypeout{}%
\makeatletter
\if@twoside\lthtmltypeout{latex2htmlLength evensidemargin=\the\evensidemargin}%
\else\lthtmltypeout{latex2htmlLength evensidemargin=\the\oddsidemargin}\fi%
\lthtmltypeout{}%
\makeatother
\setcounter{page}{1}
\onecolumn

% !!! IMAGES START HERE !!!

\setcounter{secnumdepth}{3}
\setcounter{tocdepth}{3}

\renewcommand{\contentsname}{\'Indice}
\stepcounter{chapter}
\stepcounter{section}
\stepcounter{subsection}
\stepcounter{subsection}
\stepcounter{section}
\stepcounter{section}
\stepcounter{subsection}
\stepcounter{subsection}
\stepcounter{subsection}
\stepcounter{subsection}
{\newpage\clearpage
\lthtmlinlinemathA{tex2html_wrap_inline2491}%
$ (O_2)$%
\lthtmlindisplaymathZ
\lthtmlcheckvsize\clearpage}

{\newpage\clearpage
\lthtmlinlinemathA{tex2html_wrap_inline2493}%
$ (CO_2)$%
\lthtmlindisplaymathZ
\lthtmlcheckvsize\clearpage}

\stepcounter{subsection}
\stepcounter{subsubsection}
\stepcounter{subsubsection}
\stepcounter{subsubsection}
\stepcounter{section}
\stepcounter{subsection}
\stepcounter{subsection}
\stepcounter{subsection}
\stepcounter{subsection}
\stepcounter{subsection}
\stepcounter{chapter}
\stepcounter{section}
\stepcounter{subsection}
\stepcounter{subsection}
{\newpage\clearpage
\lthtmlpictureA{tex2html_wrap2538}%
\includegraphics[width=1\linewidth]{Tema2/02_mapa-fisico-espana-normal.pdf}%
\lthtmlpictureZ
\lthtmlcheckvsize\clearpage}

\stepcounter{subsubsection}
\stepcounter{subsubsection}
\stepcounter{subsubsection}
\stepcounter{subsubsection}
\stepcounter{subsubsection}
\stepcounter{section}
\stepcounter{subsection}
\stepcounter{subsubsection}
\stepcounter{subsubsection}
\stepcounter{subsection}
\stepcounter{section}
\stepcounter{subsection}
\stepcounter{subsection}
\stepcounter{subsection}
\stepcounter{section}
\stepcounter{subsection}
\stepcounter{subsection}
\stepcounter{section}
\stepcounter{subsection}
\stepcounter{subsection}
\stepcounter{subsubsection}
\stepcounter{subsubsection}
\stepcounter{subsubsection}
\stepcounter{subsection}
\stepcounter{section}
\stepcounter{subsection}
\stepcounter{subsubsection}
\stepcounter{subsubsection}
\stepcounter{subsection}
\stepcounter{section}
\stepcounter{subsection}
\stepcounter{subsection}
\stepcounter{subsection}
\stepcounter{subsubsection}
\stepcounter{subsubsection}
\stepcounter{subsubsection}
\stepcounter{section}
\stepcounter{subsection}
\stepcounter{subsection}
\stepcounter{subsection}
\stepcounter{chapter}
\stepcounter{section}
\stepcounter{subsection}
\stepcounter{subsubsection}
\stepcounter{subsubsection}
\stepcounter{subsubsection}
\stepcounter{subsection}
\stepcounter{subsubsection}
\stepcounter{subsubsection}
\stepcounter{subsection}
\stepcounter{subsubsection}
\stepcounter{subsubsection}
\stepcounter{subsubsection}
\stepcounter{subsection}
\stepcounter{subsubsection}
\stepcounter{subsubsection}
\stepcounter{subsection}
\stepcounter{subsubsection}
\stepcounter{subsection}
\stepcounter{subsubsection}
\stepcounter{subsubsection}
\stepcounter{subsubsection}
\stepcounter{section}
\stepcounter{subsection}
\stepcounter{subsubsection}
\stepcounter{subsection}
\stepcounter{subsubsection}
\stepcounter{subsubsection}
\stepcounter{subsubsection}
\stepcounter{subsection}
\stepcounter{subsubsection}
\stepcounter{subsubsection}
\stepcounter{section}
\stepcounter{subsection}
\stepcounter{subsubsection}
\stepcounter{subsubsection}
\stepcounter{subsubsection}
\stepcounter{subsection}
\stepcounter{subsubsection}
\stepcounter{chapter}
\stepcounter{section}
\stepcounter{subsection}
\stepcounter{subsubsection}
\stepcounter{subsubsection}
\stepcounter{subsubsection}
\stepcounter{subsection}
\stepcounter{subsection}
\stepcounter{chapter}
\stepcounter{chapter}

\end{document}

\documentclass[a4paper,12pt]{article}
\usepackage[utf8]{inputenc}
\usepackage[table,xcdraw,x11names]{xcolor} % Para los colores en las celdas de la tabla y las transparencias
\usepackage{graphicx}
\usepackage{transparent} % Paquete para aplicar transparencia a imágene
\usepackage{fancyhdr}
\usepackage{geometry}
\usepackage{lastpage} % Para obtener el número total de páginas
\usepackage[absolute,overlay]{textpos} % Paquete para posiciones absolutas
\usepackage{atbegshi} % Paquete para controlar elementos en cada página
\usepackage{eso-pic} % Paquete para añadir elementos en el fondo de la página
\definecolor{verdeoscuro}{rgb}{0.0, 0.2, 0.0}
\usepackage{longtable} % Para tablas de varias páginas
\usepackage{array} % Para usar el entorno array dentro de celdas
\usepackage{multirow}
\usepackage{multicol}
\usepackage[hidelinks]{hyperref}                                     %Para añadir enlaces.
\geometry{a4paper, top=2cm, left=2cm, right=2cm, bottom=4.8cm}
\setlength{\headheight}{70pt} % Ajusta la altura del encabezado según sea necesario
\setlength{\footskip}{14.5pt} % Ajusta la distancia del pie de página al texto
% Eliminar la indentación al principio de cada párrafo
\setlength{\parindent}{0pt}
% Establecer la distancia estándar entre párrafos
\setlength{\parskip}{1.5em}

% Configuración del encabezado y pie de página
\pagestyle{fancy}
\fancyhf{}

% Encabezado personalizado
\fancyhead[L]{
    \begin{minipage}[b]{1.5cm}
        \includegraphics[height=50pt]{imagenes/logo_cole_transp.png} % Logo o imagen1 en el encabezado
    \end{minipage}
    \hspace{0.2cm} % Espacio entre logo y texto
    \begin{minipage}[b]{6cm}
        \footnotesize{\textbf{CEIP MAR MEDITERRÁNEO} \\
        Consejería de Desarrollo Educativo \\
        y Formación Profesional \\
        \vspace{0.01cm}}
    \end{minipage}
    \hspace{0.1cm} % Espacio entre el texto y la imagen europa.png
    \begin{minipage}[b]{1.5cm}
        \includegraphics[height=50pt]{imagenes/europa.png} % Imagen2 a la derecha del texto
        \vspace{0.01cm}
    \end{minipage}
    \hspace{2.5cm} % Espacio entre las dos imágenes
    \begin{minipage}[b]{2.5cm}
        \includegraphics[height=30pt]{imagenes/junta.png} % Imagen3 a la derecha de imagen2
        \vspace{-0.3cm} % Ajusta la posición vertical del número de página
        \makebox[4cm][r]{\footnotesize{Página \thepage\ de \pageref{LastPage}}} % Número de página alineado a la derecha
        \vspace{0.01cm}
    \end{minipage}
}

% Configuración del texto en el pie de página usando 'fancyhdr'
\fancyfoot[L]{
    \begin{minipage}[t]{7cm} % Minipage para controlar el área del texto
        \raggedright % Alinea el texto a la izquierda
        \rule{7cm}{0.1pt} \\
        \textcolor{verdeoscuro}{\footnotesize Calle José Morales Abad, 9. 04007 Almería} \\
        \textcolor{verdeoscuro}{Tlf.: 950 15 62 03} \\
        \textcolor{verdeoscuro}{04005326.edu@juntadeandalucia.es}
    \end{minipage}
}

% Añadir la imagen escudo.png como fondo en todas las páginas en la esquina inferior derecha
\AddToShipoutPictureBG{%
    \AtPageLowerLeft{%
        \put(\LenToUnit{\dimexpr\paperwidth-2.8cm\relax},\LenToUnit{0.2cm}){%
            \transparent{0.2}\includegraphics[height=100pt]{imagenes/escudo.png} % Ajusta el tamaño de la imagen con transparencia
        }
    }
}

\renewcommand{\contentsname}{Índice}  % Cambia 'Contents' por 'Índice'

\begin{document}

\begin{center}
    \section*{DIARIO DE CLASE}
\end{center}

\tableofcontents

\newpage

\sloppy % Para partir las palabras

\setcounter{section}{-1}  % Establece el contador de secciones en -1
\section{Repaso de cuarto}

\subsection{20240918 - Repaso de cuarto.}

\begin{enumerate}
    \item Las tablas de multiplicar.
    \item Mi primer programa en \textit{Python}.
\end{enumerate}

\subsection{20240919 - Repaso de cuarto.}

\begin{enumerate}
    \item Las tablas de multiplicar.
    \item Mi primer programa en \textit{Python}.
    \item Los números.
    \begin{itemize}
        \item Los números de seis cifras.
        \item El valor de las cifras de un número.
        \item Comparación y ordenación de números.
        \item Aproximación de números.
        \item Los números ordinales.
        \item Los números romanos.
    \end{itemize}
    \item La suma y la resta.
    \begin{itemize}
        \item La suma y sus propiedades.
        \item La resta. La prueba de la resta.
        \item El paréntesis.
        \item Cálculo estimado.
    \end{itemize}
    \item La multiplicación.
    \begin{itemize}
        \item La multiplicación y sus términos.
        \item Las tablas de multiplicar.
        \item Propiedades de la multiplicación.
        \item Propiedad distributiva.
        \item La multiplicación por una cifra.
        \item La multiplicación por varias cifras.
        \item La multiplicación por la unidad seguida de ceros.
    \end{itemize}
\end{enumerate}

\subsection{20240920 - Repaso de cuarto.}

\begin{enumerate}
    \item Las tablas de multiplicar.
    \item Mi primer programa en \textit{Python}.
    \item Estadística y azar.
    \begin{itemize}
        \item Datos cualitativos y datos cuantitativos.
        \item Tablas de registro de datos.
        \item Pictogramas.
        \item Gráficos de barras.
        \item Gráficos de líneas.
        \item Experiencias aleatorias y sucesos.
    \end{itemize}
    \item La división.
    \begin{itemize}
        \item La división como reparto.
        \item La división como partición.
        \item La prueba de la división.
        \item La división con ceros en el cociente.
        \item La división entre la unidad seguida de ceros.
        \item Operaciones combinadas.
    \end{itemize}
    \item Las fracciones.
    \begin{itemize}
        \item Las fracciones y sus términos.
        \item Medios, tercios y cuartos.
        \item Nombramos a las fraciones.
        \item Comparación de fracciones.
    \end{itemize}
\end{enumerate}

\subsection{20240925 - Repaso de cuarto}

\begin{enumerate}
    \item Prueba de nivel de las tablas de multiplicar.
    \item Los número decimales y el dinero.
    \begin{itemize}
        \item Fracción decimal y número decimal.
        \item Uso de monedas y billetes.
        \item Comparación y ordenación de decimales.
        \item Aproximación de números decimales.
        \item Suma y resta de números decimales.
        \item Operaciones con euros y céntimos.
    \end{itemize}
    \item Geometría y orientación en el plano.
    \begin{itemize}
        \item Rectas, semirectas y segmentos.
        \item Rectas paralelas y secantes.
        \item Los ángulos.
        \item Los ángulos según su abertura.
        \item El ángulo como giro.
        \item Traslación.
        \item Simetría.
        \item Simulación en el plano: coordenadas.
        \item Orientación espacial.
    \end{itemize}
\end{enumerate}

\subsection{20240926 - Repaso de cuarto}

\begin{enumerate}
    \item La medida del tiempo.
    \begin{itemize}
        \item El año: el calendario.
        \item Lustros, décadas, siglos y milenios.
        \item Días, horas, minutos y segundos.
        \item Relojes analógicos y digitales.
    \end{itemize}
    \item La medida de la longitud.
    \begin{itemize}
        \item El metro y sus múltiplos.
        \item Los submúltiplos del metro.
        \item Expresiones complejas e incomplejas.
        \item Operaciones con medidas de longitud.
    \end{itemize}
\end{enumerate}

\subsection{20240927 - Repaso de cuarto}

\begin{enumerate}
    \item La medida de la capacidad y de la masa.
    \begin{itemize}
        \item La medida de la capacidad.
        \item La medida de la masa.
        \item Expresiones complejas e incomplejas.
        \item Operaciones con capacidades y masas.
    \end{itemize}
    \item Figuras planas y cuerpos geométricos.
    \begin{itemize}
        \item Los polígonos.
        \item Clasificación de polígonos.
        \item Clasificación de triángulos.
        \item Clasificación de cuadriláteros.
        \item Introducción al área.
        \item Circunferencia y círculo.
        \item Poliedros. Prismas y pirámides.
        \item Cuerpos redondos.
    \end{itemize}
\end{enumerate}

\newpage

\section{Primer Trimestre}

\subsection{20241002 - Sesión 01 - Libro S1 y S2}

\begin{itemize}
    \item Conozo el valor posicional de cada cifra.
    \begin{itemize}
        \item Actividades 1, 2 y 3 página 3.
        \item Opcional: Juego Reagrupo números, página 4.
    \end{itemize}
    \item Estimo y mido.
    \begin{itemize}
        \item Lectura: \textit{En busca de lombrices y diamantes I}. (Página 5).
        \item Respondemos a las preguntas de la página 6.
        \item Actividades 1 y 2 página 7.
    \end{itemize}
    \item Programo en \textit{Python}.
    \begin{itemize}
        \item Uso el comando \textbf{\textit{print}} para imprimir texto por pantalla.
    \end{itemize}
\end{itemize}

\subsection{20241003 - Sesión 02 - Libro S3 y S4}

\begin{itemize}
    \item Revisión del trabajo de la sesión anterior.
    \item Represento diagramas estadísticos.
    \begin{itemize}
        \item Actividades 1 y 2 página 8; 3 página 9.
    \end{itemize}
    \item Utilizo paréntesis.
    \begin{itemize}
        \item Analizamos la toma de decisiones, página 10.
        \item Actividades 1, 2, 3 y 4 página 11.
        \item Opcional: Juego Cubo 21.
    \end{itemize}
    \item Programo en \textit{Python}.
    \begin{itemize}
        \item Uso de variables para guardar datos.
    \end{itemize}
\end{itemize}

\subsection{20241004 - Sesión 03 - Libro S5 y S6}

\begin{itemize}
    \item Revisión del trabajo de la sesión anterior.
    \item Calculo distancias.
    \begin{itemize}
        \item Actividades 1 página 13; 2 y 3 página 14.
    \end{itemize}
    \item Multiplico centenas por centenas.
    \begin{itemize}
        \item Actividades 1, 2 y 3 página 16.
    \end{itemize}
    \item Programo en \textit{Python}.
    \begin{itemize}
        \item Uso el comando \textbf{\textit{print}} para imprimir por pantalla texto mezclado con el contenido de variables.
    \end{itemize}
\end{itemize}

\subsection{20241009 - Sesión 04 - Libro S7 y S8}

\begin{itemize}
    \item Revisión del trabajo de la sesión anterior.
    \item Interpreto datos.
    \begin{itemize}
        \item Actividades 1 página 17 y 2 página 18.
    \end{itemize}
    \item Multiplico dos números naturales con más de tres cifras.
    \begin{itemize}
        \item Lectura: \textit{En busca de lombrices y diamantes II.} (Página 19).
        \item Respondemos a las preguntas de la página 20.
        \item Actividades 1 y 2 página 21.
    \end{itemize}
    \item Programo con \textit{Python.}
    \begin{itemize}
        \item ¿Puedo hacer que \textit{Python} me escriba cualquier tabla de multiplicar I?
    \end{itemize}
\end{itemize}

\subsection{20241010 - Sesión 05 - Libro S9 y S10}

\begin{itemize}
    \item Revisión del trabajo de la sesión anterior.
    \item Aplico la multiplicación.
    \begin{itemize}
        \item Actividades 1, 2 y 3 página 22.
        \item Opcional: Multiplicar con cuatro cubos, página 23.
    \end{itemize}
    \item Mido figuras rectangulares.
    \begin{itemize}
        \item Actividades 1, 2 y 3 página 25.
    \end{itemize}
    \item Programo con \textit{Python.}
    \begin{itemize}
        \item ¿Puedo hacer que \textit{Python} me escriba cualquier tabla de multiplicar II?
    \end{itemize}
\end{itemize}

\subsection{20241016 - Sesión 06 - Libro S11 y S12}

\begin{itemize}
    \item Revisión del trabajo de la sesión anterior.
    \item Interpreto restos.
    \begin{itemize}
        \item Lectura de la página 26.
        \item Actividad 1 página 27.
    \end{itemize}
    \item Lectura: \textit{En busca de lombrices-diamante III}. (páginas 28 y 29).
    \item Respondemos a las preguntas sobre la lectura de la página 29.
    \item Practico la división.
    \begin{itemize}
        \item Actividades 1, 2 y 3 página 30.
        \item Opcional: Cociente menor. (Página 31).
    \end{itemize}
    \item ¿Puedo hacer que \textit{Python} me escriba cualquier tabla de multiplicar III?
\end{itemize}

\subsection{20241017 - Sesión 07 - Libro S13 y S14}

\begin{itemize}
    \item Revisión del trabajo de la sesión anterior.
    \item Relaciono el dinero y los números decimales.
    \begin{itemize}
        \item Actividades 1 y 2 página 32; 3 y 4 página 33.
        \item Opcional: Reagrupo decimales. (Página 34).
    \end{itemize}
    \item Sumo, resto y comparo números decimales.
    \begin{itemize}
        \item Actividades 1, 2, 3 y 4 página 35; 5, 6, 7 y 8 página 36.
    \end{itemize}
    \item ¿Puedo hacer que \textit{Python} me escriba cualquier tabla de multiplicar IV?
\end{itemize}

\subsection{20241018 - Sesión 08 - Libro S15 y S16}

\begin{itemize}
    \item Lectura: \textit{En busca de lombrices-diamante IV}. (Páginas 37 y 38).
    \item Respondemos a las preguntas sobre la lectura de la página 38.
    \item Multiplico y divido decimales por potencias de 10.
    \begin{itemize}
        \item Actividades 1 y 2 página 39.
    \end{itemize}
    \item Aplico las aproximaciones.
    \begin{itemize}
        \item Actividades 1, 2, 3, 4 y 5 página 40.
    \end{itemize}
    \item ¿Puedo hacer que \textit{Python} me escriba cualquier tabla de multiplicar V?
\end{itemize}

\subsection{20241023 - Sesión 09 - Libro S13 y S14}

\begin{itemize}
    \item Revisión del trabajo de la sesión anterior.
    \item Repaso: Operaciones con decimales.
    \item Completamos las actividades que faltan por hacer de las páginas 32, 33, 35 y 36.
    \item Repasamos el bucle \textbf{\textit{for}} en \textit{Python}.
\end{itemize}

\subsection{20241024 - Sesión 10 - Libro S15 y S16}

\begin{itemize}
    \item Revisión del trabajo de la sesión anterior.
    \item Repaso: Divisiones por potencias de 10.
    \item Repaso: Aplicamos las aproximaciones.
    \item Completamos las actividades que faltan por hacer de las páginas 39 y 40.
    \item Repasamos el bucle \textbf{\textit{for}} en \textit{Python}.
\end{itemize}

\subsection{20241025 - Sesión 11 - Libro S17 y S18}

\begin{itemize}
    \item Revisión del trabajo de la sesión anterior.
    \item Lectura: \textit{En busca de lombrices-diamante V. (Páginas 41 y 42).}
    \item Respondemos a las preguntas sobre la lectura de la página 42.
    \item Relaciono unidades métricas.
    \begin{itemize}
        \item Actividades 1 página 43; 2, 3, 4 y 5 página 44.
    \end{itemize}
    \item Repaso los elementos de los polígonos.
    \begin{itemize}
        \item Actividades 1 y 2 página 45.
    \end{itemize}
\end{itemize}

\subsection{20241030 - Sesión 12 - Libro S19 y S20}

\begin{itemize}
    \item Revisión del trabajo de la sesión anterior.
    \item Lectura: \textit{La academia de superagentes (I)}. Páginas 46 y 47.
    \item Respondemos a las preguntas sobre la lectura. Página 47.
    \item Investigo con la calculadora.
    \item Multiplico decimales y números naturales.
    \begin{itemize}
        \item Actividades 1 y 2 página 49.
        \item Juego: Multiplico con decimales (opcional), página 51.
    \end{itemize}
\end{itemize}

\subsection{20241031 - Sesión 13 - Prueba escrita}

\begin{itemize}
    \item Prueba escrita sobre los contenidos vistos desde la sesión 1 hasta la sesión 18 del libro.
    \item Practicamos \textit{Python}:
    \begin{itemize}
        \item El comando \textit{print}.
        \item Variables.
        \item El bucle \textit{for}.
    \end{itemize}
\end{itemize}

\subsection{20241106 - Sesión 14 - Libro S21 y S22}

\begin{itemize}
    \item Lectura: \textit{La academia de superagenes II. (Páginas 52 y 53).}
    \item Respondemos a las preguntas de la página 53 sobre la lectura.
    \item Calculo cocientes decimales.
    \begin{itemize}
        \item Actividades 1, 2 y 3 página 54.
    \end{itemize}
    \item Sitúo coordenadas.
    \begin{itemize}
        \item Actividades 1 página 55 y 2 página 56.
    \end{itemize}
    \item Practico \textit{Python} en \textit{https://www.onlinegdb.com/}
\end{itemize}

\subsection{20241107 - Sesión 15 - Libro S23 y S24}

\begin{itemize}
    \item Revisión del trabajo de la sesión anterior.
    \item Practico con las funciones.
    \begin{itemize}
        \item Actividades 1 y 2 página 57.
        \item Opcional: Juego de las funciones de la página 58.
    \end{itemize}
    \item Hago predicciones con la calculadora.
    \begin{itemize}
        \item Actividades 1 y 2 página 59.
    \end{itemize} 
    \item Practico \textit{Python} en \textit{https://www.onlinegdb.com/}
\end{itemize}

\subsection{20241108 - Sesión 16 - Libro S25 y S26}

\begin{itemize}
    \item Revisión del trabajo de la sesión anterior.
    \item Sumo y resto números negativos.
    \begin{itemize}
        \item Actividades 2, 3 y 4 página 61.
    \end{itemize}
    \item Hallo normas de funciones.
    \begin{itemize}
        \item Actividades 1, 2 y 3 página 62; 4 y 5 página 63.
    \end{itemize}
    \item Practico \textit{Python} en \textit{https://www.onlinegdb.com/}
\end{itemize}

\subsection{20241113 - Sesión 17 - Libro S27 y S28}

\begin{itemize}
    \item Revisión del trabajo de la sesión anterior.
    \item Lectura: \textit{La academia de superagentes III}. Páginas 64 y 65.
    \item Respondemos a las preguntas sobre la lectura de la página 65.
    \item Calculo datos proporcionales.
    \begin{itemize}
        \item Actividades 1 página 66 y 2 página 67.
    \end{itemize}
    \item Interpreto diagramas.
    \begin{itemize}
        \item Actividades 1 página 68 y 2 página 69.
    \end{itemize}
    \item Planteamos y resolvemos dudas sobre lo aprendido en \textit{Python} hasta el momento.
\end{itemize}

\subsection{20241114 - Sesión 18 - Libro S29 y S30}

\begin{itemize}
    \item Revisión del trabajo de la sesión anterior.
    \item Encuentro la medida.
    \begin{itemize}
        \item Actividad 1 página 70.
        \item Opcional: \textit{Promedios con cubos}, página 71.
    \end{itemize}
    \item Sitúo puntos en los cuatro cuadrantes.
    \begin{itemize}
        \item Actividades 1 página 72, 2 y 3 página 74.
    \end{itemize}
    \item Planteamos y resolvemos dudas sobre lo aprendido en \textit{Python} hasta el momento.
\end{itemize}

\subsection{20241115 - Sesión 19 - Libro S31}

\begin{itemize}
    \item Revisión del trabajo de la sesión anterior.
    \item Conozco las funciones encadenadas.
    \begin{itemize}
        \item Actividades 1, 2, 3, 4 y 5 página 75; 6 página 76.
    \end{itemize}
    \item Planteamos y resolvemos dudas sobre lo aprendido en \textit{Python} hasta el momento.
\end{itemize}

\subsection{20241120 - Sesión 20 - Libro S32 y S33}

\begin{itemize}
    \item Revisión del trabajo de la sesión anterior.
    \item Busco funciones inversas.
    \begin{itemize}
        \item Actividades 1, 2 y 3 página 77.
    \end{itemize}
    \item Repaso las funciones.
    \begin{itemize}
        \item Actividades 1 página 78; 2 y 3 página 79.
    \end{itemize}
    \item Planteamos y resolvemos dudas sobre lo aprendido en \textit{Python} hasta el momento.
\end{itemize}

\subsection{20241121 - Sesión 21 - Libro S34 y S35}

\begin{itemize}
    \item Revisión del trabajo de la sesión anterior.
    \item Lectura: \textit{La academia de superagentes IV}. Páginas 80 y 81.
    \item Respondemos a las preguntas sobre la lectura de la página 81.
    \item Resuelvo divisiones con decimales.
    \begin{itemize}
        \item Actividades 1 y 2 página 82.
    \end{itemize}
    \item Aplico la división.
    \begin{itemize}
        \item Actividades 1 y 2 página 83.
    \end{itemize}
    \item Opcional: Juego de cubos: \textit{Hacemos operaciones (dividir)} de la página 84.
    \item Planteamos y resolvemos dudas sobre lo aprendido en \textit{Python} hasta el momento.
\end{itemize}

\subsection{20241122 - Sesión 22 - Libro S36 y S37}

\begin{itemize}
    \item Revisión del trabajo de la sesión anterior.
    \item Investigo rectas y ángulos.
    \begin{itemize}
        \item Actividades 1, 2, 3 y 4 página 85.
    \end{itemize}
    \item Descubro los elementos de la circunferencia y el círculo.
    \begin{itemize}
        \item Actividad 1 página 86.
    \end{itemize}
    \item Planteamos y resolvemos dudas sobre lo aprendido en \textit{Python} hasta el momento.
\end{itemize}

\subsection{20241127 - Sesión 23 - Prueba escrita}

\subsection{20241128 - Sesión 24 - Repaso}

\begin{itemize}
    \item La multiplicación con decimales.
    \item La división con decimales.
    \item Resolución de problemas en los que intervienen multiplicaciones con decimales.
    \item Resolución de problemas en los que intervienen divisiones con decimales.
    \item \href{https://sarcior.com/Matematicas5/RepasoT1.pdf}{Actividades de repaso del primer trimestre}.
\end{itemize}

\subsection{20241129 - Sesión 25 - Repaso}

\begin{itemize}
    \item La multiplicación con decimales.
    \item La división con decimales.
    \item Resolución de problemas en los que intervienen multiplicaciones con decimales.
    \item Resolución de problemas en los que intervienen divisiones con decimales.
    \item \href{https://sarcior.com/Matematicas5/RepasoT1.pdf}{Actividades de repaso del primer trimestre}.
\end{itemize}

\subsection{20241204 - Sesión 26 - Repaso}

\begin{itemize}
    \item Funciones y funciones encadenadas.
    \item Resolución de problemas a través de funciones.
    \item \href{https://sarcior.com/Matematicas5/RepasoT1.pdf}{Actividades de repaso del primer trimestre}.
\end{itemize}

\subsection{20241205 - Sesión 27 - Repaso}

\begin{itemize}
    \item Interpretación de gráficos estadísticos.
    \item Cálculo de la media aritmética.
    \item Líneas paralelas, perpendiculares y secantes.
    \item \href{https://sarcior.com/Matematicas5/RepasoT1.pdf}{Actividades de repaso del primer trimestre}.
\end{itemize}

\subsection{20241211 - Sesión 28 - Prueba escrita.}

\subsection{20241212 - Sesión 29 - Repaso}

\begin{itemize}
    \item Interpretación de gráficos estadísticos.
    \item Cálculo de la media aritmética.
    \item Líneas paralelas, perpendiculares y secantes.
    \item \href{https://sarcior.com/Matematicas5/RepasoT1.pdf}{Actividades de repaso del primer trimestre}.
\end{itemize}

\subsection{20241213 - Sesión 30 - Repaso}

\begin{itemize}
    \item Revisión de la prueba escrita.
    \item Revisión de las actividades de repaso.
    \item Lectura del libro \textit{Aprendiendo a programar con Python}.
    \item Repaso de los comandos que hemos aprendido de \textit{Python}.
\end{itemize}

\subsection{20241218 - Sesión 31 - Repaso}

\begin{itemize}
    \item Revisión de las actividades de repaso.
    \item Lectura del libro \textit{Aprendiendo a programar con Python}.
    \item Repaso de los comandos que hemos aprendido de \textit{Python}.
    \item \href{https://sarcior.com/Matematicas5/Python_para_niños_C01.pdf}{Python para niños. Capítulo 1}.
\end{itemize}

\subsection{20241219 - Sesión 32 - Repaso}

\begin{itemize}
    \item Revisión de las actividades de repaso.
    \item Lectura del libro \textit{Aprendiendo a programar con Python}.
    \item Repaso de los comandos que hemos aprendido de \textit{Python}.
    \item \href{https://sarcior.com/Matematicas5/Python_para_niños_C01.pdf}{Python para niños. Capítulo 1}.
    \item \href{https://sarcior.com/Matematicas5/Python_para_niños_C02.pdf}{Python para niños. Capítulo 2}.
\end{itemize}

\subsection{20241220 - Sesión 33 - Repaso}

\begin{itemize}
    \item Revisión de las actividades de repaso.
    \item Lectura del libro \textit{Aprendiendo a programar con Python}.
    \item Repaso de los comandos que hemos aprendido de \textit{Python}.
    \item \href{https://sarcior.com/Matematicas5/Python_para_niños_C01.pdf}{Python para niños. Capítulo 1}.
    \item \href{https://sarcior.com/Matematicas5/Python_para_niños_C02.pdf}{Python para niños. Capítulo 2}.
\end{itemize}

\section{Segundo Trimestre}

\subsection{20250108 - Sesión 34 - Libro S38 y S39}

\begin{itemize}
    \item Busco múltiplos comunes.
    \begin{itemize}
        \item Actividades 1, 2, 3, 4, 5, 6 y 7 página 3.
    \end{itemize}
    \item Conozco los divisores.
    \begin{itemize}
        \item Actividad 1 página 4; 2, 3 y 4 página 5.
    \end{itemize}
\end{itemize}

\subsection{20250109 - Sesión 35 - Libro S40 y S41}

\begin{itemize}
    \item Revisión del trabajo de la sesión anterior.
    \item Conozco los números primos y los compuestos.
    \item \href{esPrimo.txt}{Cómo sé si un número es primo.}
    \begin{itemize}
        \item Actividades 1, 2, 3, 4, 5 y 6 página 6.
    \end{itemize}
    \item Sumo fracciones.
    \begin{itemize}
        \item Actividades 1, 2 y 3 página 7; 4 y 5 página 8.
    \end{itemize}
\end{itemize}

\subsection{20250110 - Sesión 36 - Libro S42 y S43}

\begin{itemize}
    \item Revisión del trabajo de la sesión anterior.
    \item Encuentro fracciones de un número.
    \begin{itemize}
        \item Actividades 1, 2 y 3 página 9.
        \item (Opcional) Juego de los cubos: Fracciones de 60, página 10.
    \end{itemize}
    \item Relaciono fracciones con decimales.
    \begin{itemize}
        \item Actividades 1, 2 y 3 página 11.
        \item (Opcional) Juego de los cubos: Hasta 1, página 12.
    \end{itemize}
\end{itemize}

\subsection{20250115 - Sesión 37 - Libro S44 y S45}

\begin{itemize}
    \item Revisión del trabajo de la sesión anterior.
    \item Lectura: \textit{Poderes}, páginas 13 y 14.
    \item Respondemos alas preguntas sobre la lectura 1, 2 y 3 página 14 (opcional).
    \item Interpreto divisiones con múltiplos de 10.
    \begin{itemize}
        \item Actividades 1, 2 y 3 página 15.
    \end{itemize}
    \item Aproximo cocientes.
    \begin{itemize}
        \item Actividades 1, 2 y 3 página 16.
    \end{itemize}
\end{itemize}

\subsection{20250116 - Sesión 38 - Libro S46 y S47}

\begin{itemize}
    \item Revisión del trabajo de la sesión anterior.
    \item Utilizo cocientes aproximados.
    \begin{itemize}
        \item Actividades 1 página 17 y 18; 2 y 3 página 19.
        \item (Opcional) Juego de cubos: Selecciona cocientes, página 20.
    \end{itemize}
    \item (Opcional) Rutina de pensamiento: Palabra · Idea · Frase, página 21.
    \item Exploro ángulos y giros.
    \begin{itemize}
        \item Actividades 1 y 2 página 22.
    \end{itemize}
\end{itemize}

\subsection{20250117 - Sesión 39 - Libro S48 y S49}

\begin{itemize}
    \item Revisión del trabajo de la sesión anterior.
    \item Mido ángulos
    \begin{itemize}
        \item Podéis practicar con las actividades de la sesión que consideréis necesarias para adquirir los contenidos.
    \end{itemize}
    \item Investigo los ángulos de triángulos y cuadriláteros.
    \begin{itemize}
        \item Podéis practicar con las actividades de la sesión que consideréis necesarias para adquirir los contenidos.
    \end{itemize}
\end{itemize}

\subsection{20250122 - Sesión 40 - Libro S50 y S51}

\begin{itemize}
    \item Examino triángulos (explicación).
    \item Podéis practicar con las actividades de la sesión que consideréis necesarias para adquirir los contenidos.
    \item Construyo triángulos (explicación).
    \item Podéis practicar con las actividades de la sesión que consideréis necesarias para adquirir los contenidos.
    \item Resolución de dudas.
\end{itemize}

\subsection{20250123 - Sesión 41 - Libro S52}

\begin{itemize}
    \item Lectura opcional: \textit{Poderes mentales (II)}, página 32.
    \item Actividades opcionales sobre la lectura: 1, 2 y 3 página 33.
    \item Conozco la función identidad (explicación).
    \item Podéis practicar con las actividades de la sesión que consideréis necesarias para adquirir los contenidos.
    \item Resolución de dudas.
\end{itemize}

\subsection{20250124 - Sesión 42 - Libro S53}

\begin{itemize}
    \item Lectura opcional: \textit{Poderes mentales (III)}, página 35.
    \item Actividades opcionales 1, 2 y 3 página 36.
    \item Utilizo funciones encadenadas (explicación).
    \item Podéis practicar con las actividades de la sesión que consideréis necesarias para adquirir los contenidos.
    \item Resolución de dudas.
\end{itemize}

\subsection{20250129 - Sesión 43 - Libro S54}

\begin{itemize}
    \item Aprendo a escribir funciones (explicación).
    \item Podéis practicar con las actividades de la sesión que consideréis necesarias para adquirir los contenidos.
    \item Resolución de dudas.
\end{itemize}

\subsection{20250130 - Sesión 44 - Libro S55}

\begin{itemize}
    \item Escribo funciones encadenadas (explicación).
    \item Podéis practicar con las actividades de la sesión que consideréis necesarias para adquirir los contenidos.
    \item Resolución de dudas.
\end{itemize}

\subsection{20250131 - Día de la paz.}

\subsection{20250205 - Sesión 45 - Prueba escrita}

\begin{itemize}
    \item Desde S38 a S53 del libro 2.
\end{itemize}

\subsection{20250206 - Sesión 46 - Libro S56 y S57}

\begin{itemize}
    \item Mido la longitud de la circunferencia.
    \item Actividades opcionales para practicar de las páginas 42 y 43.
    \item Investigo rectas y circunferencias.
    \item Actividades opcionales para practicar de las páginas 44 y 45.
    \item Resolución de dudas.
\end{itemize}

\subsection{20250207 - Sesión 47 - Libro S58 y S59}

\begin{itemize}
    \item Lectura: En el planeta Pro-medio (I), páginas 46 y 47.
    \item Actividades opcionales sobre la lectura de la página 47.
    \item Busco congruencias y semejanzas.
    \item Actividades opcionales para practicar de la página 48.
    \item Aplico decimales.
    \item Actividad opcional para practicar de las páginas 49 y 50.
    \item Resolución de dudas.
\end{itemize}

\subsection{20250212 - Sesión 48 - Libro S60 y S61}

\begin{itemize}
    \item Lectura: \textit{En el planeta Pro-medio (II)}, páginas 51 y 52.
    \item Preguntas sobre la lectura, página 52.
    \item Conozco las tasas y las razones.
    \item Podéis practicar con las actividades 1 y 2, página 52.
    \item Utilizo razones.
    \item Podéis practicar con las actividades 1, páginas 55 y 56 y 2, página 56.
    \item Resolución de dudas.
\end{itemize}

\subsection{20250213 - Sesión 49 - Libro S62 y S63}

\begin{itemize}
    \item Lectura: \textit{En el planea Pro-medio (III)}, páginas 57 y 58.
    \item Preguntas sobre la lectura, página 58.
    \item Utilizo tasas.
    \item Podéis practicar con las actividades 1 y 2, página 59.
    \item Divido con tres cifras en el divisor.
    \item Podéis practicar con las actividades 1 y 2, página 60; 3, 4 y 5 página 61.
    \item Resolcución de dudas.
\end{itemize}

\subsection{20250214 - Sesión 50 - Libro S64 y S65}

\begin{itemize}
    \item Aplico la división.
    \item Podéis practicar con las actividades 1, página 62; 2 y 3, página 63.
    \item Conozco los triángulos semejantes y congruentes.
    \item Podéis practicar con las actividades 1, página 64; 2, página 65.
    \item Resolución de dudas.
\end{itemize}

\subsection{20250219 - Sesión 51 - Libro S66 y S67}

\begin{itemize}
    \item Identifico triángulos semejantes y congruentes.
    \item Podéis practicar con las actividades de las páginas 66 y 67.
    \item Conozco la traslación, el giro y la simetría.
    \item Podéis practicar con las actividades de las páginas 68 6 69.
    \item Resolución de dudas.
\end{itemize}

\subsection{20250220 - Sesión 52 - Libro S68 y S69}

\begin{itemize}
    \item Trazo ejes de simetría.
    \item Podéis practicar con las actividades de la página 70.
    \item Investigo fracciones equivalente.
    \item Podéis practicar con las actividades de las páginas 71 y 72.
    \item Resolución de dudas.
\end{itemize}

\subsection{20250221 - Sesión 53 - Libro S70 y S71}

\begin{itemize}
    \item Calculo fracciones equivalente.
    \item Podéis practicar con las actividades de las páginas 73 y 74.
    \item Comparo y ordeno fracciones.
    \item Podéis practicar con las actividades de las páginas 75 y 76.
    \item Resolución de dudas.
\end{itemize}

\end{document}